\documentclass[12pt,a4paper]{letter}
\usepackage[utf8]{inputenc}
\usepackage[portuguese]{babel}
\usepackage[T1]{fontenc}
\usepackage{amsmath}
\usepackage{amsfonts}
\usepackage{amssymb}
\usepackage{enumitem}
\usepackage[left=2cm,right=2cm,top=2cm,bottom=2cm]{geometry}
\begin{document}
\pagestyle{empty}
\begin{Huge}
\begin{center}
	REDE COLABORATIVA
\end{center}
\end{Huge}

\begin{Large}

Precisa usar TeX, C/C++, Fortran, MatLab, etc. mas não aprendeu durante a graduação?


Toda {\bf sexta feira}, às {\bf 14 horas} na {\bf sala 323}, logo após as reuniões 
do LPOO uma aula sobre algum conhecimento t\'{e}cnico.

\begin{center}
\framebox{
\parbox[c]{12cm}{Nesta sexta-feira dia {\bf 17/05} uma aula sobre
\begin{center}
	{\bf Introdução a C} e
\end{center}
\begin{center}
	{\bf Introdução ao git}. 
\end{center}}}
\end{center}
 
Possíveis tópicos:

\begin{minipage}{.4\textwidth}
\begin{itemize}[leftmargin=*]
\item	BibTeX
\item	TikZ
\item	HTML
\item	Vim
\item	GNU/LINUX básico
\item	C/C++
\end{itemize}
\end{minipage}
\begin{minipage}{.4\textwidth}
\begin{itemize}[leftmargin=*]
\item	Octave/MATLAB
\item	Python
\item	Julia
\item	Emacs
\item	Controle de versão (Git)
\item 	Fortran
\end{itemize}
\end{minipage}
\vspace{1cm}

Ao final de cada aula um mini-projeto será passado para treinar as habilidades 
ensinadas, focando sempre que possível em problemas de Matemática Aplicada.

Interessados devem entrar em contato com Daiane pelo e-mail \vspace{-0.5cm}
\begin{center} {\bf daianegferreira@gmail.com}, \end{center}\vspace{-0.5cm}
ou pelo grupo de e-mails \vspace{-0.5cm}
\begin{center} {\bf rede-colaborativa@googlegroups.com}. \end{center}\vspace{-0.5cm}
Maiores informações e cronograma pelo site \vspace{-0.5cm}
\begin{center}  {\bf www.lpoo.ime.unicamp.br/rc}. \end{center}

Obs.: Cada aluno deve trazer seu próprio equipamento (notebook).

\end{Large}

\end{document}