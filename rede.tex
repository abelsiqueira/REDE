\documentclass[a4paper,11pt]{report}
\usepackage[brazil]{babel}
\usepackage[T1]{fontenc}
\usepackage{ae}
\usepackage[utf8]{inputenc}
\usepackage[dvipsnames]{color}
\usepackage{graphicx}
\usepackage{epsfig}
\usepackage{makeidx}
\usepackage{multicol}
\usepackage{amssymb, amsmath, amsfonts}
\bibliographystyle{plain}
\topmargin		0 cm
\hoffset		0 cm
\voffset		0 cm
\evensidemargin		0 cm
\oddsidemargin		0 cm
\setlength{\textwidth}{16 cm}
\setlength{\textheight}{21 cm}
\usepackage{lpoo}

\title{ Rede de Colaboração }
\author{ Laboratório de Pesquisa Operacional e Otimização }
\date{ }

\numberwithin{equation}{section}
\newcommand{\red}{\color{red}}
\newcommand{\blue}{\color{blue}}
\newcommand{\green}{\color{green}}

\makeindex

\begin{document}
\maketitle

\chapter{Manifesto}

\section{ Rede de Colaboração }

\subsection{ Objetivo }

Proporcionar um ambiente de ensino e aprendizado colaborativo, focando 
nos alunos de pós-graduação da Matemática Aplicada, de forma que o 
investimento no ensino promova uma contribuição em outra área.

\subsection{ Possíveis Tópicos a Serem Abordados }

\begin{itemize}
  \item Comandos Básicos de Linux
  \item C/C++
  \item Fortran
  \item MatLab/Octave
  \item Git
  \item Makefile
  \item LaTeX
  \item Python
  \item Gdb
  \item Linguagem de Marcação Markdown
  \item Linguagem de Marcação RST
  \item TikZ
\end{itemize}

\subsection{ Possíveis Contribuições }

\begin{itemize}
  \item Ensino
  \item Biblioteca de Algoritmos para o Grupo de Otimização
  \item Resoluções de Exercícios (focando nas principais disciplinas)
  \item Elaboração de Apostilas
\end{itemize}

\subsection{ Horário e Local }

Sexta-feira, sala 323, às 14h, logo após as reuniões do LPOO.

\subsection{ Logística }

\begin{itemize}
  \item Os tópicos não precisam estar ligados
  \item As aulas devem, sempre que possível, utilizar alguma forma de contribuição para ensinar o assunto.
  \item As aulas, a priori, exigem que o aluno traga seu computador.
  \item A duração das aulas deve variar entre 30 e 60 minutos.
  \item O professor deve devolver o material da sala para a secretaria de pós.
\end{itemize}

\subsection{ Funções Administrativas }

\begin{itemize}
  \item {\bf Administrador {\it (Mestre da Masmorra)}:} Abel Soares Siqueira
  \item {\bf Mantenedor (site e cronograma) {\it (Guardião do Portal)}:} 
    Raniere Gaia Costa da Silva
  \item {\bf Marketing {\it (Clérigo da Ordem)}:} Bruno Henrique Cervelin
  \item {\bf Recrutador (para professores e alunos) {\it (Monge Wololoo)}:} 
    Daiane Gonçalves Ferreira
  \item {\bf Assessor Administrativo {\it (Loremaster)}:} Deise Gonçalves Ferreira
\end{itemize}

\subsection{ Descrição das Funções }

\begin{itemize}
  \item Administrador
  \begin{itemize}
    \item Organiza as aulas (Em conjunto com o professor específico);
    \item Mantém o relacionamento da equipe;
    \item Atribui as tarefas.
  \end{itemize}
  \item Mantenedor
  \begin{itemize}
    \item Atualiza o site;
    \item Atualiza o cronograma.
  \end{itemize}
  \item Marketing
  \begin{itemize}
    \item Cria material de divulgação para a Rede;
    \item Divulga o grupo pessoalmente para grupos de pessoas;
  \end{itemize}
  \item Recrutador
  \begin{itemize}
    \item Matricula os alunos;
    \item Mantém o banco de dados dos integrantes;
    \item Organiza os requisitos de aulas.
  \end{itemize}
  \item Assistente Administrativo
  \begin{itemize}
    \item Produz as atas das reuniões;
    \item Mantém o banco de documentos.
  \end{itemize}
\end{itemize}

\chapter{Banco de Integrantes}

\begin{itemize}
  \item RA (int)
  \item Nome (String)
  \item E-mail (String)
  \item Interesses (Opções fixas/Lista/Texto?)
\end{itemize}

\chapter{Estrutura das Aulas}

Toda aula deve ter um projeto/exercício que corresponde a uma realização.
Quando possível, um material de referência deve ser indicado para a matéria.
Quando possível, o projeto deve poder ser avaliado automaticamente, utilizando
scripts.
Sempre que possível, o projeto deve ser relacionado com a área de
matemática aplicada, e, quando possível, deve poder ser aproveitado para o 
LPOO.

Os projetos/exercícios, devem ser feitos em latex, para serem disponibilizados
no banco de documentos, e passados aos alunos no formato pdf. (Salvo casos
específicos).
Os documentos anexos ao projeto, devem estar disponíveis de forma avulsa,
para que o aluno possa obtê-lo facilmente e sem depender de ferramentas de
compactação, e também para melhor armazenamento no banco de documentos.

\chapter{Sistema de Realizações e Níveis}

\section{ Realizações }

As realizações são tarefas ou exercícios disponíveis para
o integrante que premiam o integrante com uma quantidade de pontos
de experiência.
Cada realização só pode ser concluída uma vez, e uma vez
que o integrante a concluiu, ele não a perde.

Cada aula deve ter ao menos uma realização, e preferencialmente
uma envolvendo a conclusão de um curso ou módulo. 

As realizações devem ser classificadas em fácil, média,
difícil e muito difícil, e cada nível de dificuldade correspondente a uma
quantidade diferente de pontos de recompensa. Além disso, cada realização
tem um símbolo para identificar a dificuldade.
\begin{center}
  \begin{tabular}{lrc} 
   Fácil         &  5  & \easy \\ 
   Médio         & 15  & \medium \\ 
   Difícil       & 30  & \hard \\ 
   Muito Difícil & 60  & \veryhard \\ 
  \end{tabular}
\end{center}

\section{ Níveis }

Ao acumular uma quantidade suficiente de pontos de experiência, o 
integrante passa para um nível diferente.

Os níveis podem ser apenas por questões estéticas, ou podemos
exigir um certo nível para participação de alguns eventos.

A quantidade de experiência para alcançar os níveis são
\begin{center}
    \begin{tabular}{|c|c|c|c|c|} \hline
    & {\bf Iniciante} & {\bf Médio} & {\bf Avançado} & {\bf Especialista} \\ \hline
    {\bf Alquimista}  &    0 &  100 &  220 &  360 \\ \hline
    {\bf Bruxo}       &  510 &  710 &  930 & 1190 \\ \hline
    {\bf Mago} \\ \hline
    {\bf Invocador} \\ \hline
    {\bf Arquimago} \\ \hline
  \end{tabular}
\end{center}
\begin{center}
  \begin{tabular}{cccc} 
    \MedalhaAlquimista{1} & \MedalhaAlquimista{2} & 
      \MedalhaAlquimista{3} & \MedalhaAlquimista{4} \\ 
    \MedalhaBruxo{1} & \MedalhaBruxo{2} & 
      \MedalhaBruxo{3} & \MedalhaBruxo{4} \\ 
    \MedalhaMago{1} & \MedalhaMago{2} & 
      \MedalhaMago{3} & \MedalhaMago{4} \\ 
    \MedalhaInvocador{1} & \MedalhaInvocador{2} & 
      \MedalhaInvocador{3} & \MedalhaInvocador{4} \\ 
    \MedalhaArquimago{1} & \MedalhaArquimago{2} & 
      \MedalhaArquimago{3} & \MedalhaArquimago{4} \\ 
  \end{tabular}
\end{center}

\end{document}
